\renewcommand{\figwidth}{0.43\textwidth}
\begin{figure}[H]
	\centering
	\begin{subfigure}[b]{\figwidth}
		\includegraphics[width=\textwidth,trim=0 0 350 0, clip]{img/Bernou_percentile_5_count.png}
		\caption{Confusion matrix of \bn.}
		\label{fig:confmat-be}
	\end{subfigure}
	~
	\begin{subfigure}[b]{\figwidth}
		\includegraphics[width=\textwidth,trim=0 0 350 0, clip]{img/Multinomial_percentile_5_count.png}
		\caption{Confusion matrix of \mn.}
		\label{fig:confmat-mn}
	\end{subfigure}
	\\
	\begin{subfigure}[b]{\figwidth}
		\includegraphics[width=\textwidth,trim=0 0 350 0, clip]{img/RandomForest_percentile_5_count.png}
		\caption{Confusion matrix of \rf.}
		\label{fig:confmat-rf}
	\end{subfigure}
	~
	\begin{subfigure}[b]{\figwidth}
		\includegraphics[width=\textwidth,trim=0 0 350 0, clip]{img/SVM_percentile_5_count.png}
		\caption{Confusion matrix of SVM.}
		\label{fig:confmat-svm}
	\end{subfigure}
	\\
	\begin{subfigure}[b]{\figwidth}
		\includegraphics[width=\textwidth,trim=0 0 350 0, clip]{img/hybrid_percentile_5_count.png}
		\caption{Confusion matrix of \hy.}
		\label{fig:confmat-hybrid}
	\end{subfigure}
	\caption{Confusion matrices for the different classifiers. A total of 231 articles were tested. A vocabulary size of 511 words and the data type \emph{"Mapped value from 0 to 1"} were used. The rightmost column describes for a certain topic how many articles it contains and the percentage of how correct the model predicted. The bottom row describes for a certain predicted class, how many times that was the target class, and the percentage correct out of them.}
	\label{fig:confmat}
\end{figure}
