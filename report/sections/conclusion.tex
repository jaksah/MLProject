\mn\ \nb\ proved to be the most suitable classifier for the task of classifying news articles of eight different classes, since it obtained the highest hit ratio and had a good overall performance for different datatypes.
\\\\
\svm\ proved to be the classifier that is most sensitive of the data type. Figures \ref{fig:hitratio} and \ref{fig:hitratio-data} show that the performance is poor for the data type count and that the performance is best for the $L_2$-normalized data. There are however many parameters, such as Kernel function, penalty parameter $C$, and tolerance for stopping, for this classifier that can be fine-tuned and maybe yield a better result.
\\\\
It would be of interest as future work to do an investigation of how the \rf\ classifier behaves if the data is evenly balanced between the classes. For further reading in the field of very imbalanced data, see \cite{Chen}.
\\\\
Adding document characteristics as features, such as average word length and number of digits, and analyzing other pruning techniques, e.g. TF-IDF which emphasizes words that are unique for a document and suppress the value of common words among all documents, would also be an interesting extension to this report.
\\\\
It would be interesting to see how the hit ratio would turn out if a human was to classify 100 articles from these eight topics. 