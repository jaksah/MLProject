Main focus of this report is to analyze, and compare between models, the performance impact of varying vocabulary size for different data types as well as difficulties of distinguishing the eight different topics. The result of the analysis can be seen in Figure \ref{fig:hitratio} \& \ref{fig:confmat} and Table \ref{tab:similarity}.\\\\
Figure \ref{fig:hitratio} shows, for the different models and data types, the overall hit ratio of correct classification as a function of vocabulary size after $\chi^2$ pruning. By comparing the figures it can be noted that the maximum accuracy of 76\% is attained with binary inputs by Multinomial at a vocabulary size of 1022 words. Another observation is that Bernoulli is sensitive to vocabulary size compared to other models and SVM seems in contrast to Bernoulli more robust to changes in vocabulary size. More results derived from these figures are that Random Forest is independent of the feature format and Hybrid follows the behavior of Multinomial.\\\\
The confusion matrices in Figure \ref{fig:confmat} describes the accuracy for specific topics. The rightmost column describes for a certain topic how many articles it contains and the percentage of how correct the model predicted. The bottom row describes for a certain predicted class, how many times that was the target class, and the percentage correct out of them. Analyzing the matrices one can observe that Sport and Education is fairly easy to classify in contrast to Technology and Education which is often misclassified as Science \& Environment and Business respectively. Notable is that Random Forest has a tendency of predicting Sports.\\\\
In Table \ref{tab:similarity} a comparison of the similarity of prediction when mispredicting is shown. The values are resembles, given that both classifiers predicted wrong class, how many times they predicted the same class.
%Table of similarities
\ifnum\printdraft>0
	Here goes specific results for Multinomial.
\else
\begin{center}
  	\textbf{--- DRAFT PARTS ---}
\end{center}
\fi


\onecolumn
\ifnum\printdraft>0
	Here goes specific results for Naïve bayes.
\else
\begin{center}
  	\textbf{--- DRAFT PARTS ---}
\end{center}
\fi
\onecolumn
\newcommand{\figwidth}{0.45\textwidth}
\begin{figure}[H]
	\centering
	\begin{subfigure}[b]{\figwidth}
		\includegraphics[width=\textwidth]{Bernoulli-hitrate.eps}
		\caption{Hit ratio of Bernoulli classifier with varying vocabulary size.}
		\label{fig:hitrate-nb}
	\end{subfigure}
	~
	\begin{subfigure}[b]{\figwidth}
		\includegraphics[width=\textwidth]{Multinomial-hitrate.eps}
		\caption{Hit ratio of Multinomial classifier with varying vocabulary size.}
		\label{fig:hitrate-mn}
	\end{subfigure}
	\\
	\begin{subfigure}[b]{\figwidth}
		\includegraphics[width=\textwidth]{Random-Forest-hitrate.eps}
		\caption{Hit ratio of Random Forest classifier with varying vocabulary size.}
		\label{fig:hitrate-rf}
	\end{subfigure}
	~
	\begin{subfigure}[b]{\figwidth}
		\includegraphics[width=\textwidth]{SVM-hitrate.eps}
		\caption{Hit ratio of SVM classifier with varying vocabulary size.}
		\label{fig:hitrate-svm}
	\end{subfigure}
	\\
	\begin{subfigure}[b]{\figwidth}
		\includegraphics[width=\textwidth]{Hybrid-hitrate.eps}
		\caption{Hit ratio of Hybrid classifier with varying vocabulary size.}
		\label{fig:hitrate-hybrid}
	\end{subfigure}
	\caption{Hit ratio vs vocabulary size}
	\label{fig:hitrate}
\end{figure}
\twocolumn

\onecolumn
\renewcommand{\figwidth}{0.43\textwidth}
\begin{figure}[H]
	\centering
	\begin{subfigure}[b]{\figwidth}
		\includegraphics[width=\textwidth,trim=0 0 350 0, clip]{img/Bernou_percentile_5_count.png}
		\caption{Confusion matrix of \bn.}
		\label{fig:confmat-be}
	\end{subfigure}
	~
	\begin{subfigure}[b]{\figwidth}
		\includegraphics[width=\textwidth,trim=0 0 350 0, clip]{img/Multinomial_percentile_5_count.png}
		\caption{Confusion matrix of \mn.}
		\label{fig:confmat-mn}
	\end{subfigure}
	\\
	\begin{subfigure}[b]{\figwidth}
		\includegraphics[width=\textwidth,trim=0 0 350 0, clip]{img/RandomForest_percentile_5_count.png}
		\caption{Confusion matrix of \rf.}
		\label{fig:confmat-rf}
	\end{subfigure}
	~
	\begin{subfigure}[b]{\figwidth}
		\includegraphics[width=\textwidth,trim=0 0 350 0, clip]{img/SVM_percentile_5_count.png}
		\caption{Confusion matrix of SVM.}
		\label{fig:confmat-svm}
	\end{subfigure}
	\\
	\begin{subfigure}[b]{\figwidth}
		\includegraphics[width=\textwidth,trim=0 0 350 0, clip]{img/hybrid_percentile_5_count.png}
		\caption{Confusion matrix of \hy.}
		\label{fig:confmat-hybrid}
	\end{subfigure}
	\caption{Confusion matrices for the different classifiers. A total of 231 articles were tested. A vocabulary size of 511 words and the data type \emph{"Mapped value from 0 to 1"} were used. The rightmost column describes for a certain topic how many articles it contains and the percentage of how correct the model predicted. The bottom row describes for a certain predicted class, how many times that was the target class, and the percentage correct out of them.}
	\label{fig:confmat}
\end{figure}

\onecolumn
\setlength\figureheight{0.25\linewidth}
\setlength\figurewidth{0.35\linewidth}
\begin{figure}[H]
	\centering
	\begin{subfigure}[b]{\figwidth}
		\tikzstyle{every node}=[font=\scriptsize]
		% This file was created by matlab2tikz v0.4.3.
% Copyright (c) 2008--2013, Nico Schlömer <nico.schloemer@gmail.com>
% All rights reserved.
% 
% The latest updates can be retrieved from
%   http://www.mathworks.com/matlabcentral/fileexchange/22022-matlab2tikz
% where you can also make suggestions and rate matlab2tikz.
% 
\begin{tikzpicture}

\begin{axis}[%
width=\figurewidth,
height=\figureheight,
scale only axis,
xmin=0,
xmax=503,
xlabel={Number of articles in training data},
ymin=0.5,
ymax=0.8,
ylabel={Hit ratio},
title={Bernoulli}
]
\addplot [
color=black,
solid,
mark=square,
mark options={solid},
forget plot
]
table[row sep=crcr]{
6 0.5325\\
11 0.6147\\
16 0.632\\
21 0.6407\\
41 0.6537\\
81 0.6753\\
161 0.6797\\
322 0.671\\
503 0.671\\
};
\addplot [
color=black,
solid,
mark=o,
mark options={solid},
forget plot
]
table[row sep=crcr]{
6 0.5325\\
11 0.6147\\
16 0.632\\
21 0.6407\\
41 0.6537\\
81 0.6753\\
161 0.6797\\
322 0.671\\
503 0.671\\
};
\addplot [
color=black,
solid,
mark=triangle,
mark options={solid,,rotate=180},
forget plot
]
table[row sep=crcr]{
6 0.5325\\
11 0.6147\\
16 0.632\\
21 0.6407\\
41 0.6537\\
81 0.6753\\
161 0.6797\\
322 0.671\\
503 0.671\\
};
\addplot [
color=black,
solid,
mark=triangle,
mark options={solid},
forget plot
]
table[row sep=crcr]{
6 0.5325\\
11 0.6147\\
16 0.632\\
21 0.6407\\
41 0.6537\\
81 0.6753\\
161 0.6797\\
322 0.671\\
503 0.671\\
};
\end{axis}
\end{tikzpicture}%
		\caption{Hit ratio of Bernoulli classifier with varying number of articles in training data. All values greater than zero is mapped to one, hence the different data types result in same accuracy.}
		\label{fig:hitratio-data-nb}
	\end{subfigure}
	~
	\begin{subfigure}[b]{\figwidth}
		\tikzstyle{every node}=[font=\scriptsize]
		% This file was created by matlab2tikz v0.4.3.
% Copyright (c) 2008--2013, Nico Schlömer <nico.schloemer@gmail.com>
% All rights reserved.
% 
% The latest updates can be retrieved from
%   http://www.mathworks.com/matlabcentral/fileexchange/22022-matlab2tikz
% where you can also make suggestions and rate matlab2tikz.
% 
\begin{tikzpicture}

\begin{axis}[%
width=\figurewidth,
height=\figureheight,
scale only axis,
xmin=0,
xmax=503,
xlabel={Number of articles in training data},
ymin=0.5,
ymax=0.8,
ylabel={Hit ratio},
title={Multinomial}
]
\addplot [
color=black,
solid,
mark=square,
mark options={solid},
forget plot
]
table[row sep=crcr]{
6 0.6667\\
11 0.7186\\
16 0.7316\\
21 0.7316\\
41 0.7446\\
81 0.7532\\
161 0.7706\\
322 0.7316\\
503 0.7229\\
};
\addplot [
color=black,
solid,
mark=o,
mark options={solid},
forget plot
]
table[row sep=crcr]{
6 0.6883\\
11 0.7013\\
16 0.7186\\
21 0.7273\\
41 0.7403\\
81 0.7186\\
161 0.7489\\
322 0.7359\\
503 0.7446\\
};
\addplot [
color=black,
solid,
mark=triangle,
mark options={solid,,rotate=180},
forget plot
]
table[row sep=crcr]{
6 0.5844\\
11 0.6407\\
16 0.6753\\
21 0.684\\
41 0.7013\\
81 0.6883\\
161 0.6883\\
322 0.6883\\
503 0.6926\\
};
\addplot [
color=black,
solid,
mark=triangle,
mark options={solid},
forget plot
]
table[row sep=crcr]{
6 0.645\\
11 0.6883\\
16 0.7359\\
21 0.7316\\
41 0.7489\\
81 0.7446\\
161 0.7359\\
322 0.7186\\
503 0.7273\\
};
\end{axis}
\end{tikzpicture}%
		\caption{Hit ratio of Multinomial classifier with varying number of articles in training data.\\\ \\\ }
		\label{fig:hitratio-data-mn}
	\end{subfigure}
	\\
	\begin{subfigure}[b]{\figwidth}
		\tikzstyle{every node}=[font=\scriptsize]
		% This file was created by matlab2tikz v0.4.3.
% Copyright (c) 2008--2013, Nico Schlömer <nico.schloemer@gmail.com>
% All rights reserved.
% 
% The latest updates can be retrieved from
%   http://www.mathworks.com/matlabcentral/fileexchange/22022-matlab2tikz
% where you can also make suggestions and rate matlab2tikz.
% 
\begin{tikzpicture}

\begin{axis}[%
width=\figurewidth,
height=\figureheight,
scale only axis,
xmin=0,
xmax=503,
xlabel={Number of articles in training data},
ymin=0.5,
ymax=0.8,
ylabel={Hit ratio},
title={Random Forest}
]
\addplot [
color=black,
solid,
mark=square,
mark options={solid},
forget plot
]
table[row sep=crcr]{
6 0.5281\\
11 0.5714\\
16 0.6017\\
21 0.619\\
41 0.6623\\
81 0.658\\
161 0.684\\
322 0.645\\
503 0.6494\\
};
\addplot [
color=black,
solid,
mark=o,
mark options={solid},
forget plot
]
table[row sep=crcr]{
6 0.5541\\
11 0.5887\\
16 0.6147\\
21 0.619\\
41 0.6623\\
81 0.6623\\
161 0.6537\\
322 0.6537\\
503 0.6364\\
};
\addplot [
color=black,
solid,
mark=triangle,
mark options={solid,,rotate=180},
forget plot
]
table[row sep=crcr]{
6 0.5368\\
11 0.5974\\
16 0.6234\\
21 0.6407\\
41 0.632\\
81 0.645\\
161 0.6364\\
322 0.6407\\
503 0.6277\\
};
\addplot [
color=black,
solid,
mark=triangle,
mark options={solid},
forget plot
]
table[row sep=crcr]{
6 0.5238\\
11 0.5931\\
16 0.6277\\
21 0.6234\\
41 0.6234\\
81 0.6537\\
161 0.645\\
322 0.6407\\
503 0.6407\\
};
\end{axis}
\end{tikzpicture}%
		\caption{Hit ratio of Random Forest classifier with varying number of articles in training data.}
		\label{fig:hitratio-data-rf}
	\end{subfigure}
	~
	\begin{subfigure}[b]{\figwidth}
		\tikzstyle{every node}=[font=\scriptsize]
		% This file was created by matlab2tikz v0.4.3.
% Copyright (c) 2008--2013, Nico Schlömer <nico.schloemer@gmail.com>
% All rights reserved.
% 
% The latest updates can be retrieved from
%   http://www.mathworks.com/matlabcentral/fileexchange/22022-matlab2tikz
% where you can also make suggestions and rate matlab2tikz.
% 
\begin{tikzpicture}

\begin{axis}[%
width=\figurewidth,
height=\figureheight,
scale only axis,
xmin=0,
xmax=503,
xlabel={Number of articles in training data},
ymin=0.5,
ymax=0.8,
ylabel={Hit ratio},
title={SVM}
]
\addplot [
color=black,
solid,
mark=square,
mark options={solid},
forget plot
]
table[row sep=crcr]{
6 0.5108\\
11 0.5238\\
16 0.6017\\
21 0.5801\\
41 0.6407\\
81 0.6667\\
161 0.6537\\
322 0.6407\\
503 0.6407\\
};
\addplot [
color=black,
solid,
mark=o,
mark options={solid},
forget plot
]
table[row sep=crcr]{
6 0.5065\\
11 0.5238\\
16 0.5887\\
21 0.6017\\
41 0.6234\\
81 0.5844\\
161 0.5584\\
322 0.5887\\
503 0.5758\\
};
\addplot [
color=black,
solid,
mark=triangle,
mark options={solid,,rotate=180},
forget plot
]
table[row sep=crcr]{
6 0.6407\\
11 0.6623\\
16 0.7273\\
21 0.7013\\
41 0.7186\\
81 0.7489\\
161 0.7532\\
322 0.7359\\
503 0.7273\\
};
\addplot [
color=black,
solid,
mark=triangle,
mark options={solid},
forget plot
]
table[row sep=crcr]{
6 0.6104\\
11 0.5974\\
16 0.658\\
21 0.6537\\
41 0.6797\\
81 0.671\\
161 0.6926\\
322 0.6667\\
503 0.6537\\
};
\end{axis}
\end{tikzpicture}%
		\caption{Hit ratio of SVM classifier with varying number of articles in training data.}
		\label{fig:hitratio-data-svm}
	\end{subfigure}
	\\
	\begin{subfigure}[b]{\figwidth}
		\tikzstyle{every node}=[font=\scriptsize]
		% This file was created by matlab2tikz v0.4.3.
% Copyright (c) 2008--2013, Nico Schlömer <nico.schloemer@gmail.com>
% All rights reserved.
% 
% The latest updates can be retrieved from
%   http://www.mathworks.com/matlabcentral/fileexchange/22022-matlab2tikz
% where you can also make suggestions and rate matlab2tikz.
% 
\begin{tikzpicture}

\begin{axis}[%
width=\figurewidth,
height=\figureheight,
scale only axis,
xmin=0,
xmax=503,
xlabel={Number of articles in training data},
ymin=0.5,
ymax=0.8,
ylabel={Hit ratio},
title={Hybrid},
legend style={at={(1.03,1)},anchor=north west,draw=black,fill=white,legend cell align=left}
]
\addplot [
color=black,
solid,
mark=square,
mark options={solid}
]
table[row sep=crcr]{
6 0.6234\\
11 0.6494\\
16 0.6883\\
21 0.7013\\
41 0.7056\\
81 0.7186\\
161 0.7273\\
322 0.7056\\
503 0.7013\\
};
\addlegendentry{Binary};

\addplot [
color=black,
solid,
mark=o,
mark options={solid}
]
table[row sep=crcr]{
6 0.6537\\
11 0.658\\
16 0.697\\
21 0.6926\\
41 0.7143\\
81 0.6883\\
161 0.71\\
322 0.7056\\
503 0.7273\\
};
\addlegendentry{Count};

\addplot [
color=black,
solid,
mark=triangle,
mark options={solid,,rotate=180}
]
table[row sep=crcr]{
6 0.5931\\
11 0.645\\
16 0.6926\\
21 0.697\\
41 0.7143\\
81 0.7229\\
161 0.7186\\
322 0.7186\\
503 0.7229\\
};
\addlegendentry{L2-normalized};

\addplot [
color=black,
solid,
mark=triangle,
mark options={solid}
]
table[row sep=crcr]{
6 0.619\\
11 0.6623\\
16 0.7143\\
21 0.7056\\
41 0.7143\\
81 0.7143\\
161 0.7316\\
322 0.7186\\
503 0.7273\\
};
\addlegendentry{0-1 mapped};

\end{axis}
\end{tikzpicture}%
		\caption{Hit ratio of Hybrid classifier with varying number of articles in training data.}
		\label{fig:hitratio-data-hybrid}
	\end{subfigure}
	\caption{Hit ratio vs number of articles in training data.}
	\label{fig:hitratio}
\end{figure}
\twocolumn
